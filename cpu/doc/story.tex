\section{越过的千难万险}

\subsection{奇偶校验}

在单独调试串口的时候,助教早已告诉了我们,要开奇校验。由于我们一直没遇到问题,所以一直都没开奇校验。突然某一天,我们的串口突然不能正常工作了,连续两条串口发来的消息总会有一条出错。我们百思不得其解,我们仔细研究了其出错的情形,寻找出错数据的共同点,最后终于发现了原因。

\subsection{奇妙的A指令}

刚开始的时候,我们的A指令总是只能写一条指令,每次写多条的时候,后面的指令都写不进去。

于是我们便开始了单步调试,单步的Term根本无法运行,我们只好使用串口模拟Term的行为,Term给串口发什么,我们就给串口发什么。我们按照源码自己解读收到的串口信息。经历了千辛万苦之后,终于尝到了甘泉。我们发现有一个地方的控制出了差错,原来是在修改指令RAM的时候,有一种情况没有插气泡。发现了这个错误,我们欣喜若狂,立即修改了这个BUG,于是我们的Term算是可以运行了。

\subsection{虚妄的加速}

我们的CPU不能运行在25M主频下。这一点很令我们头疼。

我们研究了ISE给我们的时间分析和时间报告,找到了关键路径。

我们开始报告24M,我们想办法增加到了29M,又提高到了36M,最后报告达到了46M。我们加了各种时间约束,各种约束都通过了。然而最后我们的CPU依然承受不住25M的洗礼。我认为可能是我们的板子延迟略高于标准值,或者是其它的什么原因导致的。

我们最后只能跑在21M的主频下。老师说21M和25M差不太多,我们对于旁路和分支预测的处理很好,这已经足够了。

