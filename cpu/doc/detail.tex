\section{具体实现}

\subsection{CPU模块}

CPU模块无可厚非是我们本次实现中最重要的一个模块,这个模块里面包含了非常多的原件,我们使用了如下组件来实现我们整个CPU。下面会一一列举。

\begin{center}
    \includegraphics[height=10cm]{image/detail/detail_cpu.png}
    \fcaption{CPU结构图}\label{fig:cpu_structure}
\end{center}

\subsubsection{锁存单元}

锁存单元包含if/id阶段,id/alu阶段,alu/mem阶段,mem/wb阶段四个大的锁存器,在上升沿触发。这几个锁存器的行为都受到中央控制单元的控制,中央控制单元可以命令其进行气泡的插入,以及重置功能。

这四个部件的图如\fref{fig:ifid}-\fref{fig:idalu}所示,具体信号如\tref{table:ifid}-\tref{table:idalu}所示。

\begin{center}
    \includegraphics[height=10cm]{image/detail/detail_ifid.png}
    \fcaption{IF/ID阶段锁存器设计图}\label{fig:ifid}
\end{center}

\begin{center}
    \tcaption{IF/ID阶段锁存器信号}\label{table:ifid}
    \begin{longtable}{p{0.2\columnwidth}p{0.8\columnwidth}}
        \toprule
        信号 & 信号描述 \\
        \midrule
        clk & cpu的时钟信号,上升沿的时候根据ctl\_bubble和ctl\_rst进行控制。如果ctl\_bubble和ctl\_rst均为低电平则进行锁存,将in\_pc, in\_pc\_inc, in\_instruction进行锁存并输出。 \\
        rst & 异步清空信号,由外部控制开关接入。 \\
        ctl\_bubble & 气泡控制信号,由中央控制单元给出,如果该信号为高电平则表示下一个时钟上升沿,输出数据保持不变,低电平则该控制无效。 \\
        ctl\_copy &  由中央控制单元给出,用来进行数据拷贝。\\
        ctl\_rst & 重置控制信号,由中央控制单元给出,如果如果该信号为高电平则表示下一个时钟输出清空即为一条NOP指令,低电平则该控制无效。 \\
        in\_pc & 表示下一条将要锁存的指令的pc。 \\
        in\_pc\_inc & 表示下一条将要锁存的指令的pc+1。 \\
        in\_instruction & 表示下一条将要锁存的指令内容 \\
        out\_pc & 表示已经锁存的指令的pc。 \\
        out\_pc\_inc & 表示已经锁存的指令的pc+1。 \\
        out\_instruction & 表示已经锁存的指令。 \\
        \bottomrule
    \end{longtable}
\end{center}

\begin{center}
    \includegraphics[height=10cm]{image/detail/detail_idalu.png}
    \fcaption{ID/ALU阶段锁存器设计图}\label{fig:idalu}
\end{center}

\begin{center}
    \tcaption{ID/ALU阶段锁存器信号}\label{table:idalu}
    \begin{longtable}{p{0.2\columnwidth}p{0.8\columnwidth}}
        \toprule
        信号 & 信号描述 \\
        \midrule
        in\_ra  & 这是下一条指令decode出来的alu操作数a的寄存器值,注意不是数值,会传递给中央控制单元进行旁路选择。 \\
        % 用来传输给选择器最后送到alu进行计算
        in\_rb &  这是下一条指令decode出来的alu操作数b的寄存器值,注意不是数值,会传递给中央控制单元进行旁路选择。\\
        in\_rc &  这是下一条指令decode出来的寄存器c的值,注意不是数值,会传递给中央控制单元进行旁路选择,也会传递给alumem锁存器。c的寄存器表示的
        是写回的寄存器,非常重要,所以要一直往后传。\\
        in\_data\_a &  这是下一条指令decode出来的alu操作数a的值,用来传输给选择器,(选择器可能会选择旁路),最后送到alu进行计算。 \\
        in\_data\_b &  这是下一条指令decode出来的alu操作数b的值,用来传输给选择器,(选择器可能会选择旁路),最后送到alu进行计算。 \\\\
        in\_alu\_op &  这是下一条指令alu的操作码,会传输给三个alu,具体内容请看alu部分。\\
        in\_pc &  表示下一条将要锁存的指令的pc。\\
        in\_pc\_inc &  表示下一条将要锁存的指令的pc+1。\\
        in\_imm &  表示下一条将要锁存的decode出来的立即数。\\
        in\_wr\_reg &  表示下一条指令是否需要在writeback阶段写回寄存器。\\
        in\_wr\_mem &  表示下一条指令是否需要在memory阶段写内存。\\
        in\_rd\_mem &  表示下一条指令是否需要在memory阶段读内存。\\
        in\_use\_imm & 表示下一条指令在alu阶段是否需要使用立即数,这个信号会帮助中央控制单元进行alu\_data\_b旁路的控制。 \\
        in\_alumem\_alu\_  res\_equal\_rc &  表示下一条指令到memory阶段的时候,alu的出的结果是否会在writeback阶段写回寄存器。这个信号也是为了帮助中央控制单元进行旁路控制。\\
        in\_memwb\_wb\_  alu\_mem &  表示下一条指令在writeback阶段写回的数据是memory阶段读出的数据,还是在alu阶段算出的结果。\\
        in\_is\_branch\_  except\_b & 表示下一条指令是否是branch指令,除了b指令以外的branch指令都是高电平,这个信号是帮助中央控制单元进行分支预测的检验使用的信号。 \\
        ctl\_bubble &  气泡控制信号,由中央控制单元给出,如果该信号为高电平则表示下一个时钟上升沿,输出数据保持不变,低电平则该控制无效。\\
        ctl\_copy &  由中央控制单元给出,用来进行数据拷贝。\\
        ctl\_rst &  重置控制信号,由中央控制单元给出,如果如果该信号为高电平则表示下一个时钟输出清空即为一条NOP指令,低电平则该控制无效。\\
        clk &  cpu的时钟信号,上升沿的时候根据ctl\_bubble和ctl\_rst进行控制。如果ctl\_bubble和ctl\_rst均为低电平则进行锁存,将所有带有out前轴的信号对应的in信号锁存然后输出。\\
        rst &  异步清空信号,由外部控制开关接入。\\
        out\_ra &  表示alu阶段运算的寄存器a的值,这个值会送到中央控制单元进行旁路的控制。\\
        out\_rb &  表示alu阶段运算的寄存器b的值,这个值会送到中央控制单元进行旁路的控制。\\
        out\_rc &  表示writeback阶段写回的寄存器c的值。\\
        out\_rd &  rd是一个特殊的输出,只会在sw这条指令进行使用,rd也需要送到中央控制单元进行旁路的控制,他的内容和rb完全一致。\\
        out\_data\_a &  表示alu阶段运算a的值,这个值是从一个四选一的选择器得到的,这个选择器的控制由中央控制单元给出。\\
        out\_data\_b &  表示alu阶段运算b的值,这个值是从一个四选一的选择器得到的,这个选择器的控制由中央控制单元给出。\\
        out\_data\_d &  表示memory阶段sw指令可能用到的值,这个值是从一个四选一的选择器得到的,这个选择器的控制由中央控制单元给出。\\
        out\_alu\_op &  表示当前alu进行的运算符号。\\
        out\_pc &  表示已经锁存的指令的pc。\\
        out\_pc\_inc &  表示已经锁存的指令的pc+1。\\
        out\_imm &  表示已经锁存的指令的立即数,这个数字会连接到alu\_data\_b的选择器。\\
        out\_alumem\_alu\_  res\_equal\_rc &  表示当前指令到memory阶段的时候,alu的出的结果是否会在writeback阶段写回寄存器。这个信号也是为了帮助中央控制单元进行旁路控制。\\
        out\_memwb\_wb\_  alu\_mem &  表示当前指令在writeback阶段写回的数据是memory阶段读出的数据,还是在alu阶段算出的结果。\\
        out\_is\_branch\_  except\_b &  表示当前指令是否是branch指令,除了b指令以外的branch指令都是高电平,这个信号是帮助中央控制单元进行分支预测的检验使用的信号。\\
        out\_wr\_reg &  表示当前指令是否需要在writeback阶段写回寄存器。\\
        out\_wr\_mem &  表示当前指令是否需要在memory阶段写内存。\\
        out\_rd\_mem &  表示当前指令是否需要在memory阶段读内存。\\
        out\_use\_imm & 表示当前指令在alu阶段是否需要使用立即数,这个信号会帮助中央控制单元进行alu\_data\_b旁路的控制。 \\
        \bottomrule
    \end{longtable}
\end{center}

\begin{center}
    \includegraphics[height=10cm]{image/detail/detail_alumem.png}
    \fcaption{ALU/MEM阶段锁存器设计图}\label{fig:alumem}
\end{center}

\begin{center}
    \tcaption{ALU/MEM阶段锁存器信号}\label{table:alumem}
    \begin{longtable}{p{0.2\columnwidth}p{0.8\columnwidth}}
        \toprule
        信号 & 信号描述 \\
        \midrule
        in\_rc &  这是下一条指令decode出来的寄存器c的值,注意不是数值,会传递给中央控制单元进行旁路选择,也会传递给memwb锁存器。c的寄存器表示的
        是写回的寄存器,非常重要,所以要一直往后传。\\
        in\_rd & 这是专门为了sw指令设计的寄存器的值。\\ 
        in\_data\_rd & 这是专门为了sw指令设计的寄存器的值,通过一个4️选1选择可以确保数据的正确性。由于我们的设计问题,alu在进行sw指令的时候无法将sw的源寄存器值进行旁路计算解决数据冲突,所以使用了单独的一条旁路来解决这个问题。\\
        in\_pc &  表示下一条将要锁存的指令的pc。\\
        in\_pc\_inc &  表示下一条将要锁存的指令的pc+1。\\
        in\_wr\_reg &  表示下一条指令是否需要在writeback阶段写回寄存器。\\
        in\_wr\_mem &  表示下一条指令是否需要在memory阶段写内存。\\
        in\_rd\_mem &  表示下一条指令是否需要在memory阶段读内存。\\
        in\_alu\_res & 表示下一条指令alu得出的结果。\\
        
        
        in\_alumem\_alu\_  res\_equal\_rc & 表示下一条指令alu计算出的结果是否会在writeback阶段写回寄存器。这个信号也是为了帮助中央控制单元进行旁路控制。\\
        in\_memwb\_wb\_  alu\_mem & 表示下一条指令在writeback阶段写回的数据是memory阶段读出的数据,还是在alu阶段算出的结果。\\
        clk & cpu的时钟信号,上升沿的时候根据ctl\_bubble和ctl\_rst进行控制。如果ctl\_bubble和ctl\_rst均为低电平则进行锁存,将所有带有out前轴的信号对应的in信号锁存然后输出。\\
        rst & 异步清空信号,由外部控制开关接入。\\
        ctl\_bubble &  气泡控制信号,由中央控制单元给出,如果该信号为高电平则表示下一个时钟上升沿,输出数据保持不变,低电平则该控制无效。\\
        ctl\_copy &  由中央控制单元给出,用来进行数据拷贝。\\
        ctl\_rst &  重置控制信号,由中央控制单元给出,如果如果该信号为高电平则表示下一个时钟输出清空即为一条NOP指令,低电平则该控制无效。\\
        out\_pc &  表示已经锁存的指令的pc。\\
        out\_pc\_inc &  表示已经锁存的指令的pc+1。\\
        out\_alu\_res & 表示当前指令在alu阶段通过alu计算出来的值。\\
        out\_rc & 表示当前指令decode出来的目的寄存器c的值。\\
        out\_wr\_reg &  表示当前指令是否需要在writeback阶段写回寄存器。\\
        out\_wr\_mem &  表示当前指令是否需要在memory阶段写内存。\\
        out\_rd\_mem &  表示当前指令是否需要在memory阶段读内存。\\
        out\_memwb\_wb\_  alu\_mem & 表示当前指令在writeback阶段写回的数据是memory阶段读出的数据,还是在alu阶段算出的结果。\\
        \bottomrule
    \end{longtable}
\end{center}

\begin{center}
    \includegraphics[height=10cm]{image/detail/detail_memwb.png}
    \fcaption{MEM/WB阶段锁存器设计图}\label{fig:memwb}
\end{center}

\begin{center}
    \tcaption{MEM/WB阶段锁存器信号}\label{table:memwb}
    \begin{longtable}{p{0.2\columnwidth}p{0.8\columnwidth}}
        \toprule
        信号 & 信号描述 \\
        \midrule
        clk & cpu的时钟信号,上升沿的时候根据ctl\_bubble和ctl\_rst进行控制。如果ctl\_bubble和ctl\_rst均为低电平则进行锁存,将所有带有out前轴的信号对应的in信号锁存然后输出。\\
        rst & 异步清空信号,由外部控制开关接入。\\
        ctl\_bubble &  气泡控制信号,由中央控制单元给出,如果该信号为高电平则表示下一个时钟上升沿,输出数据保持不变,低电平则该控制无效。\\
        ctl\_copy &  由中央控制单元给出,用来进行数据拷贝。\\
        ctl\_rst &  重置控制信号,由中央控制单元给出,如果如果该信号为高电平则表示下一个时钟输出清空即为一条NOP指令,低电平则该控制无效。\\
        in\_alu\_res & 表示下一条指令alu阶段计算出来的结果,可能用于写回。\\
        in\_rc & 表示下一条指令写回的寄存器值。\\
        in\_wr\_reg & 表示下一条指令是否写回寄存器。\\
        in\_mem\_res & 表示下一条指令memory阶段得到的结果。\\
        in\_memwb\_wb\_ alu\_mem & 表示下一条指令写回寄存器堆的是alu计算的结果还是memory访存的结果。\\
        out\_wr\_reg & 当前指令用来控制寄存器堆的写使能信号,使其可以在下降沿的时候更新。\\
        out\_memwb\_wb\_ alu\_mem & 当前指令写回内容2选1的控制信号,是选择alu计算的结果还是memory访存的结果。\\
        \bottomrule
    \end{longtable}
\end{center}

\subsubsection{PC锁存单元}
    PC锁存器,用来锁存PC,保证PC的改写受到中央控制部分的控制。我们规定在下降沿写入,组合逻辑输出。
    具体信号请看下表。

\begin{center}
    \includegraphics[height=10cm]{image/detail/detail_pc.png}
    \fcaption{PC锁存器}\label{fig:pc}
\end{center}
\begin{center}
    \tcaption{PC锁存器}\label{table:pc}
    \begin{longtable}{p{0.2\columnwidth}p{0.8\columnwidth}}
        \toprule
        信号 & 信号描述 \\
        \midrule
            input & 表示修改信号的输入,即修改的值。\\
            clk & CPU时钟。\\
            wr & 修改使能,如果为高电平则在下降沿修改pc的锁存器的值。\\
            rst & 异步清空信号,由外部控制开关接入。\\
            output & 当前pc锁存的值,组合逻辑。\\
        \bottomrule
    \end{longtable}
\end{center}

下一条PC的选择是一个非常复杂的结构,具体的原理图如下。
\begin{center}
    \includegraphics[height=10cm]{image/detail/detail_pc_structure.png}
    \fcaption{PC结构图}\label{fig:pcstructure}
\end{center}
具体而言,PC的选择由两条路线决定。
\begin{enumerate}
    \item 在decode阶段进行的跳转,包含jr指令以及分支预测。这里使用了图中下部分的部件,4选1的选择器可以选择pc+1,pc+1+imm,reg的值进行跳转。控制信号都是中央控制单元给出的。
    \item 在alu发现分支预测错误或者进行中断的跳转,则在上部分,最右边是一个2选1,选择进行跳转的基础地址是alu阶段的记录下来的pc地址,或者是中断的指令的地址,左边的2选1则是是否加上立即数的选择器,最后连向最左边的2选1选择器。
\end{enumerate}
综合上面的结构,构成了我们的pc整个模块,是的cpu的跳转可以正常的工作。
由于大部分控制信号都是有中央控制单元给出,控制信号将统一在中央控制器里面描述。

\subsubsection{跳转单元}
    跳转单元,用来在decode阶段进行计算和控制跳转的目的地址,这里面有旁路来处理JR指令的数据冲突。
    具体信号请看下表。
\begin{center}
    \includegraphics[height=10cm]{image/detail/detail_predict.png}
    \fcaption{跳转单元}\label{fig:predict}
\end{center}
\begin{center}
    \tcaption{跳转单元}\label{table:predict}
    \begin{longtable}{p{0.2\columnwidth}p{0.8\columnwidth}}
        \toprule
        信号 & 信号描述 \\
        \midrule
            rst & 异步清空信号,由外部控制开关接入。\\
            in\_is\_jump & 由decode给出的当前指令是不是jump指令。\\
            in\_is\_b & 由decode给出的当前指令是不是b指令。\\
            in\_is\_branch\_  except\_b & 由decode给出的当前指令是不是除了b的branch指令。 \\
            in\_predict\_res & 由中央控制单元给出的分支预测的结果。\\
            in\_jump\_reg & 由decode给出的jr指令的寄存器的值。\\
            in\_jump\_reg\_  data & 由寄存器堆给出的jr指令的寄存器的数据。\\
            in\_idalu\_alu\_  res\_equal\_rc & 这是为旁路设计的,用来计算alu阶段出现的数据冲突,这个信号表示alu阶段alu的值会在最后写回c寄存器。\\
            in\_idalu\_rc & 当前alu阶段c寄存器的值。\\
            in\_alu\_res & 当前alu阶段alu的值。\\
            in\_alumem\_rc & 当前alu阶段c寄存器的值。\\
            in\_alumem\_alu\_  res\_equal\_rc & 这是为旁路设计的,用来计算mem阶段出现的数据冲突,这个信号表示mem阶段alu的值会在最后写回c寄存器。\\
            in\_alumem\_alu\_  res & 当前memory阶段alu的值。\\
            in\_memwb\_rc & 当前writeback写回的寄存器c的值。\\
            in\_memwb\_alume  m\_res\_equal\_rc & 这是为旁路设计的,用来计算mem阶段出现的数据冲突,这个信号表示writeback阶段alu或者mem的值会在最后写回c寄存器。\\
            in\_memwb\_alume  m\_res & writeback阶段写回的值。\\
            in\_branch\_imm & 这个数据表示branch指令立即数的值,将会给pc模块用来进行加法。\\
            out\_jump\_reg\_  data & 这个输出用来连接到pc模块里面的4选1选择器,表示jr指令的目的地值。\\
            out\_branch\_imm & 这个数据用来输出表示branch指令立即数的值,将会给pc模块用来进行加法。\\
            out\_ctl\_predict & 这个数据用来输出表示分支预测的结果,连接到选择器上。\\
        \bottomrule
    \end{longtable}
\end{center}
    

\subsubsection{寄存器堆}
    寄存器堆用来输出寄存器的值,以及在时钟下降沿提供修改寄存器值的功能。
    具体信号请看下表。
\begin{center}
    \includegraphics[height=10cm]{image/detail/detail_register.png}
    \fcaption{寄存器堆}\label{fig:register}
\end{center}
\begin{center}
    \tcaption{寄存器堆}\label{table:register}
    \begin{longtable}{p{0.2\columnwidth}p{0.8\columnwidth}}
        \toprule
        信号 & 信号描述 \\
        \midrule
            clk & CPU时钟信号。\\
            rst &  异步清空信号,由外部控制开关接入。\\
            wr & 寄存器堆写使能,如果为高电平则在clk下降沿的时候将data\_c写入对应的addr\_c里面。\\
            addr\_a & 接受decode出来的寄存器a。\\
            addr\_b & 接受decode出来的寄存器b。\\
            addr\_c & 接受writeback将要写回的寄存器c。\\
            data\_a & 输出decode出来的寄存器a的数值。\\
            data\_b & 输出decode出来的寄存器b的数值。\\
            data\_c & 写回阶段修改寄存器的值。\\
        \bottomrule
    \end{longtable}
\end{center}


\subsubsection{ALU}
    我们的ALU有一点不同,我们使用了三个ALU。
    Alu,Alu\_adds,Alu\_equal,分别是三个ALU。我们将ALU区分成三个的主要原因是为了减小延时。具体原因是因为Branch指令在ALU阶段受到数据冲突旁路的影响,可能导致延时非常大。而这里的瓶颈是中央控制单元的控制信号。
    中央控制单元修改Alu的数据选择的信号的延迟会非常大,而这个操作主要过程如下。
    \begin{enumerate}
        \item 中央控制单元由于受到memory和writeback阶段锁存的信号,修改Alu的data\_b的控制信号,也就是旁路的控制。
        \item Alu进行计算,结果返回中央控制单元。
        \item 中央控制单元检查是否和分支预测出现错误如果出现错误修改控制信号。
    \end{enumerate}
    具体内容请看下表。

\begin{center}
    \includegraphics[height=10cm]{image/detail/detail_alu.png}
    \fcaption{算数逻辑单元}\label{fig:alu}
\end{center}
\begin{center}
    \tcaption{算数逻辑单元}\label{table:alu}
    \begin{longtable}{p{0.2\columnwidth}p{0.8\columnwidth}}
        \toprule
        信号 & 信号描述 \\
        \midrule
            in\_data\_a & alu的两个操作数之一。\\
            in\_data\_b & alu的两个操作数之一。\\
            in\_op & alu的操作码。\\
            out\_alu\_res & alu计算的结果。\\
        \bottomrule
    \end{longtable}
\end{center}


    为了处理数据冲突,in\_data\_a的输入是一个4选1的选择器,它有三种可能,控制信号由中央控制单元给出。
    \begin{enumerate}
        \item 寄存器堆给出的值,也就是alu阶段锁存器里面的值。
        \item memory阶段的alu计算结果,如果写回的寄存器和data\_a的寄存器一致的话。
        \item writeback阶段的alu或者memory得到的结果,如果写回的寄存器和data\_a的寄存器一致的话。
    \end{enumerate}


    同样,in\_data\_b的输入是一个4选1的选择器,它有四种可能,控制信号由中央控制单元给出。
    \begin{enumerate}
        \item 寄存器堆给出的值,也就是alu阶段锁存器里面的值。
        \item memory阶段的alu计算结果,如果写回的寄存器和data\_a的寄存器一致的话。
        \item writeback阶段的alu或者memory得到的结果,如果写回的寄存器和data\_a的寄存器一致的话。
        \item decode出来的立即数。
    \end{enumerate}

    ALU支持如下几种运算。
    \begin{enumerate}
        \item Alu\_adds支持加法,操作码为0000,为unsigned(A+B)。
        \item Alu支持减法,操作码为0001,为unsigned(A-B)。
        \item Alu支持逻辑左移,操作码为0011,为A<<B。
        \item Alu支持算数右移,操作码为0100,为A>>B。
        \item Alu支持异或,操作码为0101,为A  xor  B。
        \item Alu支持比较,操作码为0110,为A  ==  B。
        \item Alu支持符号比较,操作码为0111,为A  <=  B。
        \item Alu支持无符号比较,操作码为0111,为unsigned(A)  <=  unsigned(B)。
        \item Alu支持直接输出A,操作码为1000。
        \item Alu支持直接输出B,操作码为1001。
        \item Alu支持not,操作码为1010,为not  A。
        \item Alu\_equal支持和0比操作,操作码为1011,为A == 0。
        \item Alu\_equal支持和0比操作,操作码为1100,为A != 0。
        \item Alu支持not,操作码为1010,为not  A。
        \item Alu支持or,操作码为1011,为A  or  B。
        \item Alu支持and,操作码为1100,为A  and  B。
    \end{enumerate}

\subsubsection{译码器}
Decode用于翻译指令,生成一系列数据,并将数据传输给alu的锁存器或者跳转的控制器。
具体信号请看下表。

\begin{center}
    \includegraphics[height=10cm]{image/detail/detail_decode.png}
    \fcaption{译码器}\label{fig:decode}
\end{center}
\begin{center}
    \tcaption{译码器}\label{table:decode}
    \begin{longtable}{p{0.2\columnwidth}p{0.8\columnwidth}}
        \toprule
        信号 & 信号描述 \\
        \midrule
            in\_instruction & 输入的指令,用来进行decode。\\
            in\_pc\_inc & 当前指令的pc+1,mfpc这条指令就会使用pc+1当做立即数进行计算。\\
            out\_ra & 当前指令decode出来的寄存器a,用来进行alu计算。\\
            out\_rb & 当前指令decode出来的寄存器b,用来进行alu计算。\\
            out\_rc & 当前指令decode出来的寄存器c,当做目的寄存器,会在writeback里面进行写回。\\
            out\_ctl\_write\_reg & 控制信号,如果是高电平则表示在writeback阶段需要对rc写回。\\
            out\_ctl\_write\_mem & 控制信号,如果是高电平则表示在memory阶段需要写ram。\\
            out\_ctl\_read\_mem & 控制信号,如果是高电平则表示在memory阶段需要读ram。\\
            out\_ctl\_alu\_op & alu阶段alu的操作码。\\
            out\_use\_imm & alu阶段,数据b是否使用立即数,这个信号是辅助中央控制单元进行控制的。\\
            out\_imm & 当前指令decode出来的立即数,需要经过extend模块进行扩展。\\
            out\_ctl\_imm\_extend\_size & 控制信号,表示extend模块如何对立即数进行扩展,扩展立即数的大小。\\
            out\_ctl\_imm\_extend\_type & 控制信号,表示extend模块如何对立即数进行扩展,扩展立即数是有符号还是无符号。\\
            out\_ctl\_is\_jump & 控制指令,表示当前指令是不是j指令,传递到跳转模块。\\
            out\_ctl\_is\_b & 控制指令,表示当前指令是不是b指令,传递到跳转模块。\\
            out\_ctl\_is\_branch\_except\_b & 控制指令,表示当前指令是不是branch指令而不是b指令,传递到跳转模块和中央控制模块。\\
            out\_alumem\_alu\_res\_equal\_rc & 控制信号,表示这条指令到memory阶段时候,alu的结果是不是会在writeback阶段写回寄存器c。这个信号是帮助中央控制单元进行旁路的控制的\\
            out\_memwb\_wb\_alu\_mem & 控制信号,表示这条指令到writeback阶段时候,写回寄存器的数据是alu的结果还是访存的结果。\\
            out\_brk\_return & 这是中断返回的控制信号,传递给中央控制单元进行pc的跳转控制。\\
        \bottomrule
    \end{longtable}
\end{center}

\subsubsection{中央控制单元}
中央控制单元是最复杂的单元,里面包含了分支预测,中断的处理,以及所有控制信号的生成。这里先给出所有中央控制单元的信号。
\begin{center}
    \includegraphics[height=25cm]{image/detail/detail_center_controllor.png}
    \fcaption{中央控制单元}\label{fig:center_controllor}
\end{center}
\begin{center}
    \tcaption{中央控制单元}\label{table:center_controllor}
    \begin{longtable}{p{0.2\columnwidth}p{0.8\columnwidth}}
        \toprule
        信号 & 信号描述 \\
        \midrule
            in\_decode\_ra & 这个是decode出来ra寄存器的值,这里如果发现decode出来的ra和已经在alu的指令有冲突,即alu的指令是lw,并且写回的寄存器是ra,那么这里会产生一系列控制,进行气泡的插入。\\
            in\_decode\_rb & 这个是decode出来rb寄存器的值,这里如果发现decode出来的rb和已经在alu的指令有冲突,即alu的指令是lw,并且写回的寄存器是ra,那么这里会产生一系列控制,进行气泡的插入。\\
            in\_decode\_is\_jump & 这个信号表示decode出来的指令是不是jump指令,用处是,如果现在在memory阶段读写了instruction ram,则需要进行控制,控制的内容是保持ifid这个锁存器,并清除idalu的锁存器,具体原因是因为我们的设计导致读写instruction memory的时候pc给出的指令会出现错误,所以必须插气泡保证正确性。\\
            in\_decode\_is\_branch\_except\_b & 这个信号表示decode出来的指令是不是branch指令,用处是,如果现在在memory阶段读写了instruction ram,则需要进行控制,控制的内容是保持ifid这个锁存器,并清除idalu的锁存器,具体原因是因为我们的设计导致读写instruction memory的时候pc给出的指令会出现错误,所以必须插气泡保证正确性。\\
            in\_decode\_is\_b & 这个信号表示decode出来的指令是不是b指令,用处是,如果现在在memory阶段读写了instruction ram,则需要进行控制,控制的内容是保持ifid这个锁存器,并清除idalu的锁存器,具体原因是因为我们的设计导致读写instruction memory的时候pc给出的指令会出现错误,所以必须插气泡保证正确性。\\
            in\_idalu\_rd\_mem & 这个信号是alu阶段的指令是在memory读内存,用来帮助进行气泡控制的计算。\\
            in\_idalu\_wr\_mem & 这个信号是alu阶段的指令是在memory写内存,用来帮助进行气泡控制的计算。\\
            in\_idalu\_ra & 这个信号是alu阶段的指令的ra值,用来计算旁路的选择器。\\
            in\_idalu\_rb & 这个信号是alu阶段的指令的rb值,用来计算旁路的选择器。\\
            in\_idalu\_rc & 这个信号是alu阶段的指令的rc值,用来计算旁路的选择器。\\
            in\_idalu\_rd & 这个信号是alu阶段的指令的rd值,用来计算旁路的选择器。\\
            in\_idalu\_use\_imm\_ry & 这个是alu阶段的指令,在decode的时候生成的信号,用来告诉中央控制单元是否要使用立即数,如果使用,那么alu\_data\_b就是立即数的值而不是寄存器堆里面读出来的值了。\\
            in\_idalu\_alu\_op & 这个是alu阶段的alu操作码,根据这个生成控制信号,控制三个alu的输出里面选择哪一个送到后续的模块里面。\\
            in\_alu\_add\_res & 这个是alu\_adds的结果,用来进行是否读写instruction memory的判断,如果是instruction memory的话需要进行气泡的插入。\\
            in\_alu\_equal\_res & 个是alu\_equal的结果,用来校验branch指令的结果是不是和分支预测的结果一致。\\
            in\_idalu\_is\_branch\_except\_b & 这个是alu阶段指令是不是branch指令,告诉中央控制单元是否进行分支预测的校验。\\
            in\_alumem\_alu\_res & 这个是memory阶段alu的结果,用来帮助判断是否现在在读写instruction memory。\\ 
            in\_alumem\_rc & 这个是memory阶段的rc,用来帮助判断alu两个数据a和b的旁路的选择。\\
            in\_alumem\_wr\_mem & 这个是memory阶段是否写内存的控制信号,用来帮助判断是否现在在读写instruction memory。\\
            in\_alumem\_rd\_mem & 这个是memory阶段是否读内存的控制信号,用来帮助判断是否现在在读写instruction memory。\\
            in\_alumem\_alu\_res\_equal\_rc & 这个是memory阶段的alu的结果是否会被写回寄存器的控制信号,用来帮助判断alu两个数据a和b的旁路的选择。\\
            in\_memwb\_rc & 这个是writeback阶段写回寄存器的值,用来帮助判断alu两个数据a和b的旁路的选择。\\
            in\_memwb\_wr\_reg & 这个是writeback阶段的alu或者memory的结果是否会被写回寄存器的控制信号,用来帮助判断alu两个数据a和b的旁路的选择。\\
            in\_brk\_come & 这个是中断的开始信号,具体中断的处理方法见后面。\\
            in\_ifid\_pc & 这个是ifid锁存器的pc,用来进行中断的跳回地址的计算。\\
            in\_idalu\_pc & 这个是idalu锁存器的pc,用来进行中断的跳回地址的计算。\\
            in\_brk\_return & 这个是中断结束的信号。用来帮助中央控制单元进行pc的调整,跳回。\\
            clk & CPU的时钟。\\
            rst & 异步清空信号,由外部控制开关接入。\\
            out\_bubble\_ifid & 这个是ifid锁存器的气泡控制信号。\\
            out\_bubble\_idalu & 这个是idalu锁存器的气泡控制信号。\\
            out\_bubble\_alumem & 这个是alumem锁存器的气泡控制信号。\\
            out\_bubble\_memwb & 这个是memwb锁存器的气泡控制信号。\\
            out\_rst\_ifid & 这个是ifid锁存器的重置控制信号。\\
            out\_rst\_idalu & 这个是idalu锁存器的重置控制信号。\\
            out\_rst\_alumem & 这个是alumem锁存器的重置控制信号。\\
            out\_rst\_memwb & 这个是memwb锁存器的重置控制信号。\\
            out\_forward\_alu\_a & 这个是alu阶段数据a的4选1的控制信号。\\
            out\_forward\_alu\_b & 这个是alu阶段数据b的4选1的控制信号。\\
            out\_forward\_alu\_d & 这个是alu阶段数据d的4选1的控制信号。\\
            out\_predict\_err & 这个是pc模块里面最后一层2选1的控制信号,如果出现错误下一个pc则不由跳转模块决定,由中央控制模块决定,反之亦然。\\
            out\_predict\_res & 这个是输出给跳转单元的分支预测结果。\\
            out\_branch\_alu\_pc\_imm & 这个是pc模块里面分支预测错误以后,alu阶段里面branch指令的立即数大小,用来进行新的pc的计算。\\
            out\_pc\_wr & 这个是pc的写使能。\\
        \bottomrule
    \end{longtable}
\end{center}

\subsubsection{旁路}
我们将旁路的选择设计在中央控制器,以及decode阶段的跳转模块,从而保证数据冲突的气泡仅仅会访存出现,极大地降低了气泡的插入数量。

在alu阶段,具体的旁路技术为使用memory和writeback阶段锁存下来的内容和当前在alu的数据进行比较,如果发现冲突则进行选择。
具体的比较方法是
\begin{enumerate}
        \item 如果data\b需要使用立即数,则控制选择立即数。
        \item 如果第一条不符合在memory阶段,如果说Alu计算的结果是将要写回寄存器,并且写回的寄存器和当前alu阶段将要使用的寄存器值一样,则选择memory阶段Alu计算的结果。
        \item 如果第一条第二条不符合,则如果在writeback阶段出现了和memory阶段一样的情况按照上述类似处理。
        \item 否则选取从寄存器堆里面获得的数据。        
\end{enumerate}
而在跳转模块里面的流程修改为。
\begin{enumerate}
        \item 如果说Alu计算的结果是将要写回寄存器,并且jr的寄存器和当前alu阶段将要使用的寄存器值一样,则选择alu阶段Alu计算的结果。
        \item 如果第一条不符合,且Alu计算的结果是将要写回寄存器,并且jr的寄存器和当前memory阶段将要使用的寄存器值一样,则选择alu阶段Alu计算的结果。
        \item 如果如果第一条第二条不符合,且Alu或者memory的结果是将要写回寄存器,并且jr的寄存器和当前writeback阶段将要使用的寄存器值一样,则选择writeback阶段Alu或者memory的结果。
        \item 否则选取从寄存器堆里面获得的数据。        
\end{enumerate}

\subsubsection{分支预测}
这里还要讲一下分支预测的实现,分支预测的实现方式是维护一个大小为3的表,每个表记录指令的pc以及上一次跳转的结果。预测的过程如下。
\begin{enumerate}
        \item 如果在表中可以查到当前跳转的指令,则返回表中的答案。
        \item 否则猜测为跳转。
\end{enumerate}
更新则是在指令到了alu阶段,下降沿的时候触发更新。更新的方法为。
\begin{enumerate}
        \item 如果在表中可以查到当前跳转的指令,则更新表中的答案。
        \item 如果查不到则  表1  =  表2  表2  =  表3  表3  =  新的pc,更新表。
\end{enumerate}
这样我们实现了一个类似删除最旧表的优化,这样的效果非常好。

实验表明,有了分支预测明显减少了气泡的产生,在testbench中,加入了分支预测对于第三个测例,从时间上看,几乎没有插气泡。
\subsubsection{气泡控制}

我们的气泡插得很少,都是非常必要的地方才会插气泡。

\begin{center}
    \tcaption{气泡控制}\label{table:bubble}
    \begin{longtable}{p{0.2\columnwidth}p{0.8\columnwidth}}
        \toprule
        气泡 & 气泡描述 \\
        \midrule
        数据冲突 & 当MEM段取出寄存器的值之后,但是还未写回时,下一条指令需要使用此寄存器的值作为ALU的源操作数,我们会在decode的时候发现,插一个气泡\\
        结构冲突 & 当我们读写指令的时候,由于指令RAM总线不够,我们不能取PC对应的指令,此时我们锁住PC,并且插一个气泡\\
        分支预测 & 当分支预测错误时,我们会把错误取出的指令清空,相当于插一个气泡\\
        中断处理 & 当中断来临时,我们会让MEM和WB段继续,而清空ID和ALU段的指令,相当于插一个气泡\\
        \bottomrule
    \end{longtable}
\end{center}

除了上述描述的单个冲突以外,我们还处理了它们的组合。比如结构冲突和数据冲突同时发生的时候,我们会在两个地方插两个气泡,行为完全等同上述行为的叠加,这也是可能导致CPU效率最低的情形。