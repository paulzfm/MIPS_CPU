\section{具体实现}

\subsection{CPU模块}

CPU模块无可厚非是我们本次实现中最重要的一个模块,这个模块里面包含了非常多的原件,我们使用了如下组件来实现我们整个CPU。下面会一一列举。

\subsubsection{锁存单元}

锁存单元包含if/id阶段,id/alu阶段,alu/mem阶段,mem/wb阶段四个大的锁存器,在上升沿触发。这几个锁存器的行为都受到中央控制单元的控制,中央控制单元可以命令其进行气泡的插入,以及重置功能。

这四个部件的图如\fref{fig:ifid}-\fref{fig:idalu}所示,具体信号如\tref{table:ifid}-\tref{table:idalu}所示。

\begin{center}
    \includegraphics[height=10cm]{image/detail/detail_ifid.png}
    \fcaption{IF/ID阶段锁存器设计图}\label{fig:ifid}
\end{center}

\begin{center}
    \tcaption{IF/ID阶段锁存器信号}\label{table:ifid}
    \begin{longtable}{p{0.2\columnwidth}p{0.8\columnwidth}}
        \toprule
        信号 & 信号描述 \\
        \midrule
        clk & cpu的时钟信号,上升沿的时候根据ctl\_bubble和ctl\_rst进行控制。如果ctl\_bubble和ctl\_rst均为低电平则进行锁存,将in\_pc, in\_pc\_inc, in\_instruction进行锁存并输出。 \\
        rst & 异步清空信号,由外部控制开关接入。 \\
        ctl\_bubble & 气泡控制信号,由中央控制单元给出,如果该信号为高电平则表示下一个时钟上升沿,输出数据保持不变,低电平则该控制无效。 \\
        ctl\_copy &  由中央控制单元给出,用来进行数据拷贝。\\
        ctl\_rst & 重置控制信号,由中央控制单元给出,如果如果该信号为高电平则表示下一个时钟输出清空即为一条NOP指令,低电平则该控制无效。 \\
        in\_pc & 表示下一条将要锁存的指令的pc。 \\
        in\_pc\_inc & 表示下一条将要锁存的指令的pc+1。 \\
        in\_instruction & 表示下一条将要锁存的指令内容 \\
        out\_pc & 表示已经锁存的指令的pc。 \\
        out\_pc\_inc & 表示已经锁存的指令的pc+1。 \\
        out\_instruction & 表示已经锁存的指令。 \\
        \bottomrule
    \end{longtable}
\end{center}

\begin{center}
    \includegraphics[height=10cm]{image/detail/detail_idalu.png}
    \fcaption{ID/ALU阶段锁存器设计图}\label{fig:idalu}
\end{center}

\begin{center}
    \tcaption{ID/ALU阶段锁存器信号}\label{table:idalu}
    \begin{longtable}{p{0.2\columnwidth}p{0.8\columnwidth}}
        \toprule
        信号 & 信号描述 \\
        \midrule
        in\_ra  & 这是decode出来的alu操作数a的寄存器值,注意不是数值,会传递给中央控制单元进行旁路选择。 \\
        % 用来传输给选择器最后送到alu进行计算
        in\_rb &  这是decode出来的alu操作数b的寄存器值,注意不是数值,会传递给中央控制单元进行旁路选择。\\
        in\_rc &  这是decode出来的alu操作数c的寄存器值,注意不是数值,会传递给中央控制单元进行旁路选择,也会传递给alumem锁存器。c的寄存器表示的
        是写回的寄存器,非常重要,所以要一直往后传。\\
        in\_data\_a &  这是decode出来的alu操作数a的值,用来传输给选择器,(选择器可能会选择旁路),最后送到alu进行计算。 \\
        in\_data\_b &  这是decode出来的alu操作数b的值,用来传输给选择器,(选择器可能会选择旁路),最后送到alu进行计算。 \\\\
        in\_alu\_op &  这是alu的操作码,会传输给三个alu,具体内容请看alu部分。\\
        in\_pc &  表示下一条将要锁存的指令的pc。\\
        in\_pc\_inc &  表示下一条将要锁存的指令的pc+1。\\
        in\_imm &  表示下一条将要锁存的decode出来的立即数。\\
        in\_wr\_reg &  \\
        in\_wr\_mem &  \\
        in\_rd\_mem &  \\
        in\_use\_imm &  \\
        in\_alumem\_alu\_res\_equal\_rc &  \\
        in\_memwb\_wb\_alu\_mem &  \\
        in\_is\_branch\_except\_b &  \\
        ctl\_bubble &  \\
        ctl\_copy &  \\
        ctl\_rst &  \\
        clk &  \\
        rst &  \\
        out\_ra &  \\
        out\_rb &  \\
        out\_rc &  \\
        out\_rd &  \\
        out\_data\_a &  \\
        out\_data\_b &  \\
        out\_data\_d &  \\
        out\_alu\_op &  \\
        out\_pc &  \\
        out\_pc\_inc &  \\
        out\_imm &  \\
        out\_alumem\_alu\_res\_equal\_rc &  \\
        out\_memwb\_wb\_alu\_mem &  \\
        out\_is\_branch\_except\_b &  \\
        out\_wr\_reg &  \\
        out\_wr\_mem &  \\
        out\_rd\_mem &  \\
        out\_use\_imm &  \\
        \bottomrule
    \end{longtable}
\end{center}

\subsubsection{PC锁存单元}
    PC锁存器,用来锁存PC,保证PC的改写受到中央控制部分的控制。我们规定在下降沿写入,组合逻辑输出。
    具体信号请看下表。
