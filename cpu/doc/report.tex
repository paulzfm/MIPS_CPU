%!TEX program = xelatex
\documentclass[11pt, a4paper, titlepage]{article}

\usepackage{amsmath}
\usepackage{amssymb}

% fonts
\usepackage{xeCJK}
\setCJKmainfont[BoldFont=SimHei]{SimSun}
\setCJKfamilyfont{hei}{SimHei}
\setCJKfamilyfont{kai}{KaiTi}
\setCJKfamilyfont{fang}{FangSong}
\newcommand{\hei}{\CJKfamily{hei}}
\newcommand{\kai}{\CJKfamily{kai}}
\newcommand{\fang}{\CJKfamily{fang}}

% style
\usepackage[top=2.54cm, bottom=2.54cm, left=3.18cm, right=3.18cm]{geometry}
\linespread{1.5}
\usepackage{indentfirst}
\parindent 2em
\punctstyle{quanjiao}
\renewcommand{\today}{\number\year 年 \number\month 月 \number\day 日}

% figures and tables
\usepackage{graphicx}
\usepackage[font={bf, footnotesize}, textfont=md]{caption}
\makeatletter
    \newcommand\fcaption{\def\@captype{figure}\caption}
    \newcommand\tcaption{\def\@captype{table}\caption}
\makeatother
\usepackage{booktabs}
\renewcommand\figurename{图}
\renewcommand\tablename{表}
\newcommand{\fref}[1]{\textbf{图 \ref{#1}}}
\newcommand{\tref}[1]{\textbf{表 \ref{#1}}}
\newcommand{\tabincell}[2]{\begin{tabular}{@{}#1@{}}#2\end{tabular}} % multiply lines in one grid
\usepackage{longtable} % long table

\usepackage{listings}
\lstset{basicstyle=\ttfamily}

\usepackage{xcolor}
\renewcommand{\r}{\color{red}}
\usepackage{tabulary}
\usepackage{url}
\usepackage{hyperref}

% start of document
\title{\textbf{MIPS-16E流水CPU实验报告}}
\author{
    \kai 钱迪晨 \quad 计35 \quad 2013011402 \\
    \kai 叶子鹏 \quad 计35 \quad 2013011404 \\
    \kai 朱俸民 \quad 计35 \quad 2012011894
}
\date{\kai\today}

% -----------------start here------------------%
\begin{document}

\maketitle

\renewcommand{\contentsname}{目录}
\tableofcontents

\newpage

\section{概述}

我们实现了一个无延迟槽的带动态分支预测的流水线CPU。我们实现了一个通用性极佳(支持软硬件中断,支持绘图)的计算机。

我们最初的目的是尽可能不插,少插气泡,由于某些冲突必须暂停流水线,我们最终仅在3种情况插1个气泡,而绝大多数时候我们都无需插气泡。我们的CPU的主频最高可以达到21M(我们使用了ISE自带的模拟器件IP核DCM来分频)对于老师给的5个测例,我们花费的时间都非常少,因为我们几乎不用插气泡,所以花费的时间约等于指令数除以主频。

为了提高运行效率,我们增加了分支预测功能,branch指令和jump指令均在decode阶段进行,从而使得因为跳转引入的气泡尽可能的少。分支预测采取的是一个大小为3的查询表,每次使用pc进行查询,如果出现一次错误则更新,即记录上一次的结果。

除了CPU的核心以外,我们做了硬件中断,软件中断,像素映射的VGA接口的显示器(由于片内的RAM容量不大,不足以存下RGB,我们的显示是蓝白的)。

由于我们是像素映射,硬件的接口不仅单一而且不利于编程(非常繁琐,由于屏幕非常大,不足以用16位表示坐标,我们得传2次参数才能确定一个点),我们用软件实现了如下几个画图的接口:(详见第六节)

\begin{enumerate}
    \item 画一个点
    \item 画一条细线段
    \item 画一条粗线段
    \item 画一个等腰直角三角形
    \item 从数据RAM中读取一个形状(用于实现字符集,比如Unicode或者ASCII)
\end{enumerate}

\section{特色}


\begin{enumerate}
    \item \textbf{多条旁路}
    \item \textbf{动态分支预测}
    \item \textbf{消除延迟槽}:我们所有的跳转指令都不需要延迟槽,也不会暂停流水线
    \item \textbf{内存统一编址}
    \item \textbf{外设·键盘}:一个支持字母,数字,空格,退格,回车,shift组合键的键盘
    \item \textbf{支持硬件中断}:支持键盘触发硬件中断,为其他设备触发硬件中断留下了接口
    \item \textbf{支持软件中断}:支持软件中断scanf
    \item \textbf{外设·屏幕}:像素映射的屏幕,我们提供了字符和图形的库
    \item \textbf{支持绘图}:比字符映射的屏幕更通用
    \item \textbf{库支持}:我们用汇编写了总共千行的代码,提供了丰富的库
    \item \textbf{专用的Term}:我们为自己打造了专用的Term,方便调试,更方便展示
\end{enumerate}
\section{具体实现}

\subsection{CPU模块}

CPU模块无可厚非是我们本次实现中最重要的一个模块,这个模块里面包含了非常多的原件,我们使用了如下组件来实现我们整个CPU。下面会一一列举。

\begin{center}
    \includegraphics[height=10cm]{image/detail/detail_cpu.png}
    \fcaption{CPU结构图}\label{fig:cpu_structure}
\end{center}

\subsubsection{锁存单元}

锁存单元包含if/id阶段,id/alu阶段,alu/mem阶段,mem/wb阶段四个大的锁存器,在上升沿触发。这几个锁存器的行为都受到中央控制单元的控制,中央控制单元可以命令其进行气泡的插入,以及重置功能。

这四个部件的图如\fref{fig:ifid}-\fref{fig:idalu}所示,具体信号如\tref{table:ifid}-\tref{table:idalu}所示。

\begin{center}
    \includegraphics[height=10cm]{image/detail/detail_ifid.png}
    \fcaption{IF/ID阶段锁存器设计图}\label{fig:ifid}
\end{center}

\begin{center}
    \tcaption{IF/ID阶段锁存器信号}\label{table:ifid}
    \begin{longtable}{p{0.2\columnwidth}p{0.8\columnwidth}}
        \toprule
        信号 & 信号描述 \\
        \midrule
        clk & cpu的时钟信号,上升沿的时候根据ctl\_bubble和ctl\_rst进行控制。如果ctl\_bubble和ctl\_rst均为低电平则进行锁存,将in\_pc, in\_pc\_inc, in\_instruction进行锁存并输出。 \\
        rst & 异步清空信号,由外部控制开关接入。 \\
        ctl\_bubble & 气泡控制信号,由中央控制单元给出,如果该信号为高电平则表示下一个时钟上升沿,输出数据保持不变,低电平则该控制无效。 \\
        ctl\_copy &  由中央控制单元给出,用来进行数据拷贝。\\
        ctl\_rst & 重置控制信号,由中央控制单元给出,如果如果该信号为高电平则表示下一个时钟输出清空即为一条NOP指令,低电平则该控制无效。 \\
        in\_pc & 表示下一条将要锁存的指令的pc。 \\
        in\_pc\_inc & 表示下一条将要锁存的指令的pc+1。 \\
        in\_instruction & 表示下一条将要锁存的指令内容 \\
        out\_pc & 表示已经锁存的指令的pc。 \\
        out\_pc\_inc & 表示已经锁存的指令的pc+1。 \\
        out\_instruction & 表示已经锁存的指令。 \\
        \bottomrule
    \end{longtable}
\end{center}

\begin{center}
    \includegraphics[height=10cm]{image/detail/detail_idalu.png}
    \fcaption{ID/ALU阶段锁存器设计图}\label{fig:idalu}
\end{center}

\begin{center}
    \tcaption{ID/ALU阶段锁存器信号}\label{table:idalu}
    \begin{longtable}{p{0.2\columnwidth}p{0.8\columnwidth}}
        \toprule
        信号 & 信号描述 \\
        \midrule
        in\_ra  & 这是下一条指令decode出来的alu操作数a的寄存器值,注意不是数值,会传递给中央控制单元进行旁路选择。 \\
        % 用来传输给选择器最后送到alu进行计算
        in\_rb &  这是下一条指令decode出来的alu操作数b的寄存器值,注意不是数值,会传递给中央控制单元进行旁路选择。\\
        in\_rc &  这是下一条指令decode出来的寄存器c的值,注意不是数值,会传递给中央控制单元进行旁路选择,也会传递给alumem锁存器。c的寄存器表示的
        是写回的寄存器,非常重要,所以要一直往后传。\\
        in\_data\_a &  这是下一条指令decode出来的alu操作数a的值,用来传输给选择器,(选择器可能会选择旁路),最后送到alu进行计算。 \\
        in\_data\_b &  这是下一条指令decode出来的alu操作数b的值,用来传输给选择器,(选择器可能会选择旁路),最后送到alu进行计算。 \\\\
        in\_alu\_op &  这是下一条指令alu的操作码,会传输给三个alu,具体内容请看alu部分。\\
        in\_pc &  表示下一条将要锁存的指令的pc。\\
        in\_pc\_inc &  表示下一条将要锁存的指令的pc+1。\\
        in\_imm &  表示下一条将要锁存的decode出来的立即数。\\
        in\_wr\_reg &  表示下一条指令是否需要在writeback阶段写回寄存器。\\
        in\_wr\_mem &  表示下一条指令是否需要在memory阶段写内存。\\
        in\_rd\_mem &  表示下一条指令是否需要在memory阶段读内存。\\
        in\_use\_imm & 表示下一条指令在alu阶段是否需要使用立即数,这个信号会帮助中央控制单元进行alu\_data\_b旁路的控制。 \\
        in\_alumem\_alu\_  res\_equal\_rc &  表示下一条指令到memory阶段的时候,alu的出的结果是否会在writeback阶段写回寄存器。这个信号也是为了帮助中央控制单元进行旁路控制。\\
        in\_memwb\_wb\_  alu\_mem &  表示下一条指令在writeback阶段写回的数据是memory阶段读出的数据,还是在alu阶段算出的结果。\\
        in\_is\_branch\_  except\_b & 表示下一条指令是否是branch指令,除了b指令以外的branch指令都是高电平,这个信号是帮助中央控制单元进行分支预测的检验使用的信号。 \\
        ctl\_bubble &  气泡控制信号,由中央控制单元给出,如果该信号为高电平则表示下一个时钟上升沿,输出数据保持不变,低电平则该控制无效。\\
        ctl\_copy &  由中央控制单元给出,用来进行数据拷贝。\\
        ctl\_rst &  重置控制信号,由中央控制单元给出,如果如果该信号为高电平则表示下一个时钟输出清空即为一条NOP指令,低电平则该控制无效。\\
        clk &  cpu的时钟信号,上升沿的时候根据ctl\_bubble和ctl\_rst进行控制。如果ctl\_bubble和ctl\_rst均为低电平则进行锁存,将所有带有out前轴的信号对应的in信号锁存然后输出。\\
        rst &  异步清空信号,由外部控制开关接入。\\
        out\_ra &  表示alu阶段运算的寄存器a的值,这个值会送到中央控制单元进行旁路的控制。\\
        out\_rb &  表示alu阶段运算的寄存器b的值,这个值会送到中央控制单元进行旁路的控制。\\
        out\_rc &  表示writeback阶段写回的寄存器c的值。\\
        out\_rd &  rd是一个特殊的输出,只会在sw这条指令进行使用,rd也需要送到中央控制单元进行旁路的控制,他的内容和rb完全一致。\\
        out\_data\_a &  表示alu阶段运算a的值,这个值是从一个四选一的选择器得到的,这个选择器的控制由中央控制单元给出。\\
        out\_data\_b &  表示alu阶段运算b的值,这个值是从一个四选一的选择器得到的,这个选择器的控制由中央控制单元给出。\\
        out\_data\_d &  表示memory阶段sw指令可能用到的值,这个值是从一个四选一的选择器得到的,这个选择器的控制由中央控制单元给出。\\
        out\_alu\_op &  表示当前alu进行的运算符号。\\
        out\_pc &  表示已经锁存的指令的pc。\\
        out\_pc\_inc &  表示已经锁存的指令的pc+1。\\
        out\_imm &  表示已经锁存的指令的立即数,这个数字会连接到alu\_data\_b的选择器。\\
        out\_alumem\_alu\_  res\_equal\_rc &  表示当前指令到memory阶段的时候,alu的出的结果是否会在writeback阶段写回寄存器。这个信号也是为了帮助中央控制单元进行旁路控制。\\
        out\_memwb\_wb\_  alu\_mem &  表示当前指令在writeback阶段写回的数据是memory阶段读出的数据,还是在alu阶段算出的结果。\\
        out\_is\_branch\_  except\_b &  表示当前指令是否是branch指令,除了b指令以外的branch指令都是高电平,这个信号是帮助中央控制单元进行分支预测的检验使用的信号。\\
        out\_wr\_reg &  表示当前指令是否需要在writeback阶段写回寄存器。\\
        out\_wr\_mem &  表示当前指令是否需要在memory阶段写内存。\\
        out\_rd\_mem &  表示当前指令是否需要在memory阶段读内存。\\
        out\_use\_imm & 表示当前指令在alu阶段是否需要使用立即数,这个信号会帮助中央控制单元进行alu\_data\_b旁路的控制。 \\
        \bottomrule
    \end{longtable}
\end{center}

\begin{center}
    \includegraphics[height=10cm]{image/detail/detail_alumem.png}
    \fcaption{ALU/MEM阶段锁存器设计图}\label{fig:alumem}
\end{center}

\begin{center}
    \tcaption{ALU/MEM阶段锁存器信号}\label{table:alumem}
    \begin{longtable}{p{0.2\columnwidth}p{0.8\columnwidth}}
        \toprule
        信号 & 信号描述 \\
        \midrule
        in\_rc &  这是下一条指令decode出来的寄存器c的值,注意不是数值,会传递给中央控制单元进行旁路选择,也会传递给memwb锁存器。c的寄存器表示的
        是写回的寄存器,非常重要,所以要一直往后传。\\
        in\_rd & 这是专门为了sw指令设计的寄存器的值。\\ 
        in\_data\_rd & 这是专门为了sw指令设计的寄存器的值,通过一个4️选1选择可以确保数据的正确性。由于我们的设计问题,alu在进行sw指令的时候无法将sw的源寄存器值进行旁路计算解决数据冲突,所以使用了单独的一条旁路来解决这个问题。\\
        in\_pc &  表示下一条将要锁存的指令的pc。\\
        in\_pc\_inc &  表示下一条将要锁存的指令的pc+1。\\
        in\_wr\_reg &  表示下一条指令是否需要在writeback阶段写回寄存器。\\
        in\_wr\_mem &  表示下一条指令是否需要在memory阶段写内存。\\
        in\_rd\_mem &  表示下一条指令是否需要在memory阶段读内存。\\
        in\_alu\_res & 表示下一条指令alu得出的结果。\\
        
        
        in\_alumem\_alu\_  res\_equal\_rc & 表示下一条指令alu计算出的结果是否会在writeback阶段写回寄存器。这个信号也是为了帮助中央控制单元进行旁路控制。\\
        in\_memwb\_wb\_  alu\_mem & 表示下一条指令在writeback阶段写回的数据是memory阶段读出的数据,还是在alu阶段算出的结果。\\
        clk & cpu的时钟信号,上升沿的时候根据ctl\_bubble和ctl\_rst进行控制。如果ctl\_bubble和ctl\_rst均为低电平则进行锁存,将所有带有out前轴的信号对应的in信号锁存然后输出。\\
        rst & 异步清空信号,由外部控制开关接入。\\
        ctl\_bubble &  气泡控制信号,由中央控制单元给出,如果该信号为高电平则表示下一个时钟上升沿,输出数据保持不变,低电平则该控制无效。\\
        ctl\_copy &  由中央控制单元给出,用来进行数据拷贝。\\
        ctl\_rst &  重置控制信号,由中央控制单元给出,如果如果该信号为高电平则表示下一个时钟输出清空即为一条NOP指令,低电平则该控制无效。\\
        out\_pc &  表示已经锁存的指令的pc。\\
        out\_pc\_inc &  表示已经锁存的指令的pc+1。\\
        out\_alu\_res & 表示当前指令在alu阶段通过alu计算出来的值。\\
        out\_rc & 表示当前指令decode出来的目的寄存器c的值。\\
        out\_wr\_reg &  表示当前指令是否需要在writeback阶段写回寄存器。\\
        out\_wr\_mem &  表示当前指令是否需要在memory阶段写内存。\\
        out\_rd\_mem &  表示当前指令是否需要在memory阶段读内存。\\
        out\_memwb\_wb\_  alu\_mem & 表示当前指令在writeback阶段写回的数据是memory阶段读出的数据,还是在alu阶段算出的结果。\\
        \bottomrule
    \end{longtable}
\end{center}

\begin{center}
    \includegraphics[height=10cm]{image/detail/detail_memwb.png}
    \fcaption{MEM/WB阶段锁存器设计图}\label{fig:memwb}
\end{center}

\begin{center}
    \tcaption{MEM/WB阶段锁存器信号}\label{table:memwb}
    \begin{longtable}{p{0.2\columnwidth}p{0.8\columnwidth}}
        \toprule
        信号 & 信号描述 \\
        \midrule
        clk & cpu的时钟信号,上升沿的时候根据ctl\_bubble和ctl\_rst进行控制。如果ctl\_bubble和ctl\_rst均为低电平则进行锁存,将所有带有out前轴的信号对应的in信号锁存然后输出。\\
        rst & 异步清空信号,由外部控制开关接入。\\
        ctl\_bubble &  气泡控制信号,由中央控制单元给出,如果该信号为高电平则表示下一个时钟上升沿,输出数据保持不变,低电平则该控制无效。\\
        ctl\_copy &  由中央控制单元给出,用来进行数据拷贝。\\
        ctl\_rst &  重置控制信号,由中央控制单元给出,如果如果该信号为高电平则表示下一个时钟输出清空即为一条NOP指令,低电平则该控制无效。\\
        in\_alu\_res & 表示下一条指令alu阶段计算出来的结果,可能用于写回。\\
        in\_rc & 表示下一条指令写回的寄存器值。\\
        in\_wr\_reg & 表示下一条指令是否写回寄存器。\\
        in\_mem\_res & 表示下一条指令memory阶段得到的结果。\\
        in\_memwb\_wb\_ alu\_mem & 表示下一条指令写回寄存器堆的是alu计算的结果还是memory访存的结果。\\
        out\_wr\_reg & 当前指令用来控制寄存器堆的写使能信号,使其可以在下降沿的时候更新。\\
        out\_memwb\_wb\_ alu\_mem & 当前指令写回内容2选1的控制信号,是选择alu计算的结果还是memory访存的结果。\\
        \bottomrule
    \end{longtable}
\end{center}

\subsubsection{PC锁存单元}
    PC锁存器,用来锁存PC,保证PC的改写受到中央控制部分的控制。我们规定在下降沿写入,组合逻辑输出。
    具体信号请看下表。

\begin{center}
    \includegraphics[height=10cm]{image/detail/detail_pc.png}
    \fcaption{PC锁存器}\label{fig:pc}
\end{center}
\begin{center}
    \tcaption{PC锁存器}\label{table:pc}
    \begin{longtable}{p{0.2\columnwidth}p{0.8\columnwidth}}
        \toprule
        信号 & 信号描述 \\
        \midrule
            input & 表示修改信号的输入,即修改的值。\\
            clk & CPU时钟。\\
            wr & 修改使能,如果为高电平则在下降沿修改pc的锁存器的值。\\
            rst & 异步清空信号,由外部控制开关接入。\\
            output & 当前pc锁存的值,组合逻辑。\\
        \bottomrule
    \end{longtable}
\end{center}

下一条PC的选择是一个非常复杂的结构,具体的原理图如下。
\begin{center}
    \includegraphics[height=10cm]{image/detail/detail_pc_structure.png}
    \fcaption{PC结构图}\label{fig:pcstructure}
\end{center}
具体而言,PC的选择由两条路线决定。
\begin{enumerate}
    \item 在decode阶段进行的跳转,包含jr指令以及分支预测。这里使用了图中下部分的部件,4选1的选择器可以选择pc+1,pc+1+imm,reg的值进行跳转。控制信号都是中央控制单元给出的。
    \item 在alu发现分支预测错误或者进行中断的跳转,则在上部分,最右边是一个2选1,选择进行跳转的基础地址是alu阶段的记录下来的pc地址,或者是中断的指令的地址,左边的2选1则是是否加上立即数的选择器,最后连向最左边的2选1选择器。
\end{enumerate}
综合上面的结构,构成了我们的pc整个模块,是的cpu的跳转可以正常的工作。
由于大部分控制信号都是有中央控制单元给出,控制信号将统一在中央控制器里面描述。

\subsubsection{跳转单元}
    跳转单元,用来在decode阶段进行计算和控制跳转的目的地址,这里面有旁路来处理JR指令的数据冲突。
    具体信号请看下表。
\begin{center}
    \includegraphics[height=10cm]{image/detail/detail_predict.png}
    \fcaption{跳转单元}\label{fig:predict}
\end{center}
\begin{center}
    \tcaption{跳转单元}\label{table:predict}
    \begin{longtable}{p{0.2\columnwidth}p{0.8\columnwidth}}
        \toprule
        信号 & 信号描述 \\
        \midrule
            rst & 异步清空信号,由外部控制开关接入。\\
            in\_is\_jump & 由decode给出的当前指令是不是jump指令。\\
            in\_is\_b & 由decode给出的当前指令是不是b指令。\\
            in\_is\_branch\_  except\_b & 由decode给出的当前指令是不是除了b的branch指令。 \\
            in\_predict\_res & 由中央控制单元给出的分支预测的结果。\\
            in\_jump\_reg & 由decode给出的jr指令的寄存器的值。\\
            in\_jump\_reg\_  data & 由寄存器堆给出的jr指令的寄存器的数据。\\
            in\_idalu\_alu\_  res\_equal\_rc & 这是为旁路设计的,用来计算alu阶段出现的数据冲突,这个信号表示alu阶段alu的值会在最后写回c寄存器。\\
            in\_idalu\_rc & 当前alu阶段c寄存器的值。\\
            in\_alu\_res & 当前alu阶段alu的值。\\
            in\_alumem\_rc & 当前alu阶段c寄存器的值。\\
            in\_alumem\_alu\_  res\_equal\_rc & 这是为旁路设计的,用来计算mem阶段出现的数据冲突,这个信号表示mem阶段alu的值会在最后写回c寄存器。\\
            in\_alumem\_alu\_  res & 当前memory阶段alu的值。\\
            in\_memwb\_rc & 当前writeback写回的寄存器c的值。\\
            in\_memwb\_alumem  \_res\_equal\_rc & 这是为旁路设计的,用来计算mem阶段出现的数据冲突,这个信号表示writeback阶段alu或者mem的值会在最后写回c寄存器。\\
            in\_memwb\_alumem  \_res & writeback阶段写回的值。\\
            in\_branch\_imm & 这个数据表示branch指令立即数的值,将会给pc模块用来进行加法。\\
            out\_jump\_reg\_  data & 这个输出用来连接到pc模块里面的4选1选择器,表示jr指令的目的地值。\\
            out\_branch\_imm & 这个数据用来输出表示branch指令立即数的值,将会给pc模块用来进行加法。\\
            out\_ctl\_predict & 这个数据用来输出表示分支预测的结果,连接到选择器上。\\
        \bottomrule
    \end{longtable}
\end{center}
    

\subsubsection{寄存器堆}
    寄存器堆用来输出寄存器的值,以及在时钟下降沿提供修改寄存器值的功能。
    具体信号请看下表。
\begin{center}
    \includegraphics[height=10cm]{image/detail/detail_register.png}
    \fcaption{寄存器堆}\label{fig:register}
\end{center}
\begin{center}
    \tcaption{寄存器堆}\label{table:register}
    \begin{longtable}{p{0.2\columnwidth}p{0.8\columnwidth}}
        \toprule
        信号 & 信号描述 \\
        \midrule
            clk & CPU时钟信号。\\
            rst &  异步清空信号,由外部控制开关接入。\\
            wr & 寄存器堆写使能,如果为高电平则在clk下降沿的时候将data\_c写入对应的addr\_c里面。\\
            addr\_a & 接受decode出来的寄存器a。\\
            addr\_b & 接受decode出来的寄存器b。\\
            addr\_c & 接受writeback将要写回的寄存器c。\\
            data\_a & 输出decode出来的寄存器a的数值。\\
            data\_b & 输出decode出来的寄存器b的数值。\\
            data\_c & 写回阶段修改寄存器的值。\\
        \bottomrule
    \end{longtable}
\end{center}


\subsection{内存}

内存控制器(RamControllor),用来提供所有存储设备与外设接口的控制信号,是CPU与外部存储单元的唯一交互接口。为了避免取指与访存的结构冲突,我们设计了两个RAM控制器:RAM2用于映射指令内存,RAM1用于映射数据内存、屏幕显存和FIFO(对于它们的介绍详见外设部分)。对CPU来说,所有的存储单元均采用统一编码的地址(称\textbf{逻辑地址})进行访问,其与物理地址或外部设备的映射关系如\tref{table:mem_addr}所示。地址线为16位。

\begin{center}
    \tcaption{内存地址空间映射}\label{table:mem_addr}
    \begin{longtable}{ll}
        \toprule
        逻辑地址 & 映射到的物理地址或设备 \\
        \midrule
        0x0000 $\thicksim$ 0x7FFF & RAM2 0x0000 $\thicksim$ 0x7FFF \\
        0x8000 $\thicksim$ 0xBEFF & RAM1 0x0000 $\thicksim$ 0x3EFF \\
        0xBF00 & 串口控制信号 \\
        0xBF01 & 串口数据(低8位有效) \\
        0xBF02 & 串口2控制信号(未使用) \\
        0xBF03 & 串口2数据(未使用) \\
        0xBF04 & FIFO1读取数据 \\
        0xBF05 & FIFO2控制信号(队列是否为空) \\
        0xBF06 & FIFO2读取数据 \\
        0xBF07 & FIFO2写入数据 \\
        0xBF08 & 显存写入地址高16位 \\
        0xBF09 & 高3位为显存写入地址低3位,最低位为数据位 \\
        0xBF0A $\thicksim$ 0xBF0F & 保留 \\
        0xBF10 $\thicksim$ 0xFFFF & RAM1 0x3F10 $\thicksim$ 0x7FFF \\
        \bottomrule
    \end{longtable}
\end{center}

各设备访问时序如下:

% http://wavedrom.com/editor.html
% {signal: [
%   {name: 'data', wave: 'x2..|z.x2..', data: ['write data', 'read data']},
%   {name: 'addr', wave: 'x2..|x2...x', data: ['write addr', 'read addr']},
%   {name: 'OE', wave: '1...|0.....'},
%   {name: 'WE', wave: '1.01|......'},
%   {name: 'EN', wave: '0...|......'},
% ]}

\begin{center}
    \includegraphics[width=11cm]{image/device/sram}
    \fcaption{SRAM访问时序}
\end{center}

% {signal: [
%   {name: 'data', wave: 'x2..|zx2.z', data: ['write data', 'read data']},
%   {name: 'wrn', wave: '101.|.....'},
%   {name: 'rdn', wave: '1...|.0..1'}
% ]}

\begin{center}
    \includegraphics[width=11cm]{image/device/serial}
    \fcaption{串口访问时序}
\end{center}

% {signal: [
%   {name: 'clk', wave: '01|010'},
%   {name: 'data', wave: '2.|x..', data: ['write data']},
%   {name: 'addr', wave: '2.|2..', data: ['write addr', 'read addr']},
%   {name: 'dout', wave: 'x.|x2.', data: ['read data']},
%   {name: 'wr_en', wave: '1.|0..'},
%   {name: 'rd_en', wave: '0.|1..'}
% ]}

\begin{center}
    \includegraphics[width=8cm]{image/device/vga}
    \fcaption{显存访问时序(上升沿触发)}
\end{center}

% {signal: [
%   {name: 'clk', wave: '01|010'},
%   {name: 'data', wave: '2.|x..', data: ['write data']},
%   {name: 'dout', wave: 'x.|x2.', data: ['read data']},
%   {name: 'wr_en', wave: '1.|0..'},
%   {name: 'rd_en', wave: '0.|1..'}
% ]}

\begin{center}
    \includegraphics[width=8cm]{image/device/fifo}
    \fcaption{FIFO访问时序(上升沿触发)}
\end{center}

控制器采用两倍于CPU主频的时钟下降沿触发,分别在两个状态$S_1$与$S_2$之前跳转。对于不同访存的需求,各状态更新的控制信号如\tref{table:mem_signals}所示。初始状态下,OE, WE, rdn, wrn均为1,使能均关闭。

\begin{center}
    \tcaption{内存控制器状态转换}\label{table:mem_signals}
    \begin{longtable}{p{0.15\columnwidth}p{0.05\columnwidth}p{0.4\columnwidth}p{0.4\columnwidth}}
        \toprule
        设备 & 读/写 & $S_1$ & $S_2$ \\
        \midrule
        SRAM & 读 & OE置0,WE置1,数据线置高阻,地址线置待读取地址 & 锁存读出的数据 \\
        SRAM & 写 & OE置1,WE置1,数据线置待写入数据,地址线置待写入地址 & WE置0 \\
        串口控制 & 读 & - & 锁存输出的控制信号 \\
        串口数据 & 读 & rdn置0,wrn置1,数据线置高阻 & 锁存读出的数据,rdn置1 \\
        串口数据 & 写 & rdn置0,wrn置0,数据线置待写入数据 & wrn置0 \\
        显存地址 & 写 & 锁存地址高16位 & - \\
        显存数据 & 写 & 准备地址和数据,开启写使能,写时钟置0 & 写时钟置1 \\
        FIFO测试 & 读 & - & 锁存输出的控制信号 \\
        FIFO & 读 & 开启读使能 & 锁存读出的数据,关闭读使能 \\
        FIFO & 写 & 准备地址和数据,开启写使能,写时钟置0 & 写时钟置1 \\
        \bottomrule
    \end{longtable}
\end{center}

其中,显存与FIFO采取异步写,同步读的方式。FIFO的读时钟与控制器同频率,但是采用上升沿触发。该部件的控制信号如\tref{table:memory_ctl}所示。

\begin{center}
    \tcaption{内存控制器信号}\label{table:memory_ctl}
    \begin{longtable}{p{0.3\columnwidth}p{0.7\columnwidth}}
        \toprule
        信号 & 信号描述 \\
        \midrule
        clk & 控制器时钟。 \\
        rst & 异步清空信号,由外部控制开关接入。 \\
        in\_pc\_addr & 指令地址。 \\
        in\_ram\_addr & 访存地址。 \\
        in\_data & 写入数据。 \\
        in\_rd & 读使能。 \\
        in\_wr & 写使能。 \\
        out\_data & 读出数据。 \\
        out\_pc\_ins & 取出指令。 \\
        slow\_clk & CPU时钟,用于初始状态同步。 \\
        ram2\_oe & RAM2 OE控制输出。 \\
        ram2\_we & RAM2 WE控制输出。 \\
        ram2\_en & RAM2 EN控制输出。 \\
        ram2\_addr & RAM2 地址线输出。 \\
        ram2\_data & RAM2 数据总线。 \\
        ram1\_oe & RAM1 OE控制输出。 \\
        ram1\_we & RAM1 WE控制输出。 \\
        ram1\_en & RAM1 EN控制输出。 \\
        ram1\_addr & RAM1 地址线输出。 \\
        ram1\_data & RAM1 数据总线。 \\
        serial\_rdn & 串口rdn控制。 \\
        serial\_wrn & 串口wrn控制。 \\
        serial\_data\_ready & 串口data\_ready输入。 \\
        serial\_tbre & 串口tbre输入。 \\
        serial\_tsre & 串口tsre输入。 \\
        vga\_data & 显存写入数据。 \\
        vga\_addr & 显存写入地址。 \\
        vga\_data\_clk & 显存写时钟。 \\
        fifo1\_rd\_en & FIFO1读使能。 \\
        fifo1\_data & FIFO1读出数据。 \\
        fifo2\_rd\_en & FIFO2读使能。 \\
        fifo2\_wr\_clk & FIFO2写时钟。 \\
        fifo2\_data\_in & FIFO2写入数据。 \\
        fifo2\_data\_out & FIFO2读出数据。 \\
        fifo2\_is\_empty & FIFO2控制输入(是否为空)。 \\
        \bottomrule
    \end{longtable}
\end{center}

\section{外设和中断}

\subsection{概述}

\subsubsection{VGA像素映射的屏幕}

\begin{center}
    \includegraphics[height=10cm]{image/extension/tri.JPG}
    \fcaption{漂亮的三角形阵}\label{fig:tri}
\end{center}

屏幕为像素映射。我们在片内开了一块等同于屏幕可显示像素数目大小的RAM作为我们的显存(使用ISE的IP核)。我们只需修改显存即可改变屏幕上的内容,由于片内存储空间不大,所以只能显示蓝白两色。

像素映射为我们的计算机增添了\textbf{通用性}。

\subsubsection{键盘}

\begin{center}
    \includegraphics[height=10cm]{image/extension/35.JPG}
    \fcaption{软件中断和硬件中断的结合}\label{fig:35}
\end{center}

键盘拥有2个队列:一个叫做键盘队列,给硬件中断;一个叫做软中断队列,给软件中断。键盘输入一个字符,由硬件转译成ASCII码输送给键盘队列,当队列非空并且当时不在执行中断时,发送一个硬件中断信号,执行硬件中断。由硬件中断负责把信息转译到软中断队列。

我们使用FIFO实现了软件的缓冲区,每当回车键输入的时候,硬件中断将键盘输入的值一次性发送到软件的缓冲区,软件通过缓冲区读取字符。

\begin{enumerate}
    \item 实现0~9,a~z,退格键,空格键,回车键。硬件将其转化为ASCII码传入键盘队列
    \item 实现shift组合键,可以发送大写的字母,以及0~9上方的特殊字符
    \item 手写实现键盘队列(长度为8),软中断队列(长度为16)
\end{enumerate}


\subsubsection{硬件中断}

硬件中断是结合键盘和中央控制器实现的。每当键盘被按下,硬件中断被触发,我们会等待MEM,和WB段执行完,清空ID和ALU段。之后记录返回PC,改变PC到中断处理程序。


\subsection{细节}

\subsubsection{VGA像素映射的屏幕}

我们将大小为$640 \times 480$的整个屏幕映射到我们的显存上,我们用1位(0或1)来表示颜色,所以我们的显存空间总共为19位。 

我们的显存使用的是ISE自带的IP核,由于片内存储空间仅稍大于屏幕的可显示像素数目,我们只映射2种颜色。我们的RAM是读写端口分开的,由CPU进行写数据,由VGA控制器读出数据显示到屏幕上。

VGA控制器,我们使用了上学期老师发给我们的VGA扫屏代码,我们在其之上进行修改。由于我们的显存只够显示两种颜色(蓝色和白色),我们用0表示蓝色,1表示白色,扫屏的时候通过判断值来给RGB的线赋值。这样我们即可从我们的显存中读出屏幕的值。

CPU通过总线来写显存。由于16位的字长不足以表示整个屏幕,我们使用2个字(BF08和BF09)来向显存控制器写一个像素点,第一个字表示地址的高16位,第二个字包括地址的地位和写入的数据的RGB(这里的RGB仅仅是留作对软件的接口,由于我们只能显示2种颜色,所以我们只用1位)。

由于对于我们的这种表示来说,写一个模式(一个字母或者一个图形)对软件的要求极高,我们的工作有一部分也体现在软件上。这体现了我们的硬件的通用性,也考验我们硬件的鲁棒性。


\subsubsection{键盘,键盘队列以及软中断队列}

键盘的控制器是由上学期老师发的键盘代码修改而成的。

我们的键盘控制器能把通码转成ASCII码,滤去键盘不停发送的通码,直到接收到断码为止,也就是说我们的键盘不会导致失误。除此以外,我们支持组合键(shift + 按键可以产生大写的ASCII码),空格键,回车键,退格键······

键盘队列和软中断队列都是由我们自己实现。

这个队列是我们定义的一个模块,有2个端口,一个端口用于入队(写队列),一个端口用于出队(读队列)。每个端口都有各自的时钟(异步读写)和各自的使能端(如果不使能则队列保持不变)。队列内使用触发器存储数据,使用两个计数器来计数,分别表示队首和队尾。如果队首等于队尾则队列为空,否则队列非空,此时中断队列会发出中断信号,中央控制器会响应中断。

入队过程由上升沿触发,出队过程也是上升沿触发。是否入队/出队由入队/出队使能决定。键盘队列的容量是8个字,软中断队列的容量是16个字。

\subsubsection{硬中断}
我们每次按下键盘,键盘会将通码转换成ASCII码入队。这时队列的非空信号会触发一个硬件中断(由中央控制器实现)。

此时中央控制器会响应硬件中断,这是一个复杂的过程,整个流程分为如下几步:

\begin{enumerate}
    \item 锁住PC的写使能
    \item 存下EPC
    \item 清空流水线
    \item 跳转到中断处理程序
    \item 执行中断
    \item 中断返回
\end{enumerate}

我们详细说明如何清空流水线以及如何中断返回。

确定EPC:我们会先检查ID/ALU阶段寄存器的PC值是否为全0,如果不全0,我们将选择它。否则选择IF/ID段的PC值。为什么这样取呢?因为ID/ALU段可能是气泡,如果全0肯定是气泡,我们不能使用它。而我们整个流水线中不可能有连续2个气泡存在,所以选择IF/ID是安全的。

清空流水线:由于我们有分支预测,有插气泡的可能,我们认为流水线全部清空是不合适的。因为这时如果有分支预测错误,将不能复原。我们认为让流水线流完也是不合适的,此时PC的使能被锁住,如果有跳转将不能正常跳转。我们选择了一个折衷的方法,也是MIPS提倡的方法,把MEM和WB段的指令执行完,而ID和ALU段的指令清空。这样在我们的架构下不会产生冲突,中断得以正常进行。

中断返回:我们新增了一条ERET指令,代码是  FFFF  ,这条指令是中断执行程序的特权指令,只有在中断正在执行的时候才有效。这条指令会相当于一条B  EPC  ,表明中断执行结束,需要继续执行指令。

\subsubsection{软中断}
由软件实现,其过程需要等待硬中断的触发以及处理,通过硬中断把键盘队列中的数据转移到软中断队列中,软中断得以结束。

\section{软件方面}

\subsection{针对我们像素映射实现的接口}

\begin{enumerate}
    \item 给定坐标画一个点
    \item 给定起始点,长度,类型(总共8个类型,每45度角为一类)画一条细线段
    \item 给定起始点,长度,类型(同上),粗细,画一条粗线段
    \item 给定直角顶点,大小,类型(总共8个类型,每45度角为一类)画一个等腰直角三角形
    \item 从数据RAM中读取一个形状(用于实现字符集,比如汉字或是ASCII码)
\end{enumerate}




\subsection{针对键盘实现的硬件中断和软件中断的中断程序}

\subsubsection{硬件中断}

\begin{enumerate}
    \item 从键盘输入队列中取出一个ASCII码输出到串口
    \item 从键盘输入队列中取出一个ASCII码输入到软中断队列
    \item 从键盘输入队列中取出一个ASCII码,通过调用字符显示的函数显示到屏幕
    \item 实现退格键
\end{enumerate}

\subsubsection{软件中断}

(类似于scanf()  ,  支持char 和 int)
\begin{enumerate}
    \item 从软件中断队列中读取一个ASCII码
    \item 从软件中断队列中读取一串数字,直到空格,并解析成一个INT
    \item (暂未完成)完成一个画图程序,读入画图指令,解析成参数,传给画图函数实现画图的功能,由于时间紧张,汇编代码(1000+行)已经完成,然后没有调试完全,还不能工作。
\end{enumerate}

\subsection{定制的term}

\subsubsection{概述}
由于老师提供的term只支持基础指令,而我们需要调试也需要使用扩展的指令。

\begin{center}
    \includegraphics[height=10cm]{image/extension/term}
    \fcaption{扩展的Term}\label{fig:term}
\end{center}

\subsubsection{具体内容及实现}
我们自己的Term具备老师的Term的全部功能,除此以外,我们为它增添了新的特性,包括:
\begin{enumerate}
    \item 可以识别拓展的5条指令,包括NOT,BTNEZ,SLT,JALR,JRRA
    \item 可以向RAM中写入拓展的5条指令,包括NOT,BTNEZ,SLT,JALR,JRRA
    \item 中断的支持,对ERET的写入
\end{enumerate}

实现这个term的过程并不难,因为老师提供了term的源码,我们只需要在其A指令和U指令的模块中增加我们的指令即可。



\section{测试}

\subsection{性能测试}

我们在监控程序中对所提供的5个流水线性能测试程序进行测试,得到的结果如\tref{table:per_test}所示。CPU主频为20MHz。

\begin{center}
    \tcaption{流水线性能测试结果}\label{table:per_test}
    \begin{longtable}{cccc}
        \toprule
        测试程序 & 运行时间 (s) & 指令条数 (亿) & CPI \\
        \midrule
        1 & 6.257 & 1.25 & 1.001 \\
        2 & 10.005 & 2.00 & 1.000 \\
        3 & 4.992 & 1.00 & 0.998 \\
        4 & 4.993 & 1.00 & 0.999 \\
        5 & 4.994 & 0.75 & 1.332 \\
        \bottomrule
    \end{longtable}
\end{center}

可见,除了读写指令内存需要插气泡,导致CPI增加以外,其他程序运行的CPI都接近于1,说明数据旁路对于冲突处理得很好,只有在少数情况下才需要暂停流水线。

\subsection{扩展指令测试}

我们编写了两段程序用于测试扩展指令的正确性。

\subsubsection{扩展指令测试程序1}

程序源码为

\lstset{basicstyle=\small\ttfamily, numbers=left}
\begin{lstlisting}
    LI R0 40
    SLL R0 R0 0
    ADDIU R0 0A
    LI R1 0
    LI R2 0
    LI R3 0
    LI R4 0
    JALR R0
    JR R7
    NOP
    NOT R1 R1
    NOT R2 R2
    NOT R3 R3
    NOT R4 R4
    JRRA
    NOP
\end{lstlisting}

这段程序的功能是:置R0至R4为0,然后通过JALR指令进行过程调用,跳转到0x400A,即这里的第10行,从而完成把R1至R4分别取反赋给它们自己,即变为0xFFFF。运行前后寄存器值如\fref{fig:testp1}所示,说明结果正确。

\begin{center}
    \includegraphics[width=13cm]{image/testing/p1}
    \fcaption{测试程序1运行结果}\label{fig:testp1}
\end{center}

这段程序验证了\texttt{NOT}, \texttt{JALR}, \texttt{JRRA}的正确性。

\subsubsection{扩展指令测试程序2}

程序源码为

\begin{lstlisting}
    LI R0 1
    LI R1 2
    SLT R0 R1
    BTNEZ 5
    NOP
    LI R3 80
    JR R7
    NOP
    LI R5 0
    NOT R5 R5
    SLT R1 R0
    BTNEZ FA
    NOP
    JR R7
    NOP
\end{lstlisting}

这段程序的功能是:置R0为1,置R1为2。第3行处由于R0<R1,置T为1。由于T不等于0,第4行的跳转会被执行。即从第9行开始执行,R5先被置0,然后取反,结果为0xFFFF。之后由于R1>R0,故第12行的跳转不会执行,直接退出用户程序,返回监控程序。假设第12行跳转执行,那么在第6行R3会被赋值为0x80。运行前后寄存器值如\fref{fig:testp2}所示,R3的值未改变,而R5变成0xFFFF,说明结果正确。

\begin{center}
    \includegraphics[width=13cm]{image/testing/p2}
    \fcaption{测试程序2运行结果}\label{fig:testp2}
\end{center}

这段程序验证了\texttt{SLT}, \texttt{BTNEZ}的正确性。

\subsection{外设与中断测试}

这部分我们采用了一个绘图程序用来展示VGA的像素映射功能与键盘输入功能。由于是硬件中断,我们可以发现在键盘输入后屏幕上会立即响应,同时已经绘制好的图形也不受影响。如\fref{fig:vga_keyboard}所示。

\begin{center}
    \includegraphics[width=13cm]{image/testing/vga_keyboard}
    \fcaption{外设演示}\label{fig:vga_keyboard}
\end{center}

键盘不仅可以输入,还可以按回车键清除输入的内容。

\begin{center}
    \includegraphics[width=13cm]{image/testing/clear}
    \fcaption{按回车键清屏}
\end{center}

此外,键盘还支持大小写输入。即,键盘不仅能捕获到单个键,还能捕获到组合键,如输入大写的A则在按下SHIFT键同时按下A。

\begin{center}
    \includegraphics[width=13cm]{image/testing/case}
    \fcaption{大小写字符均可输入}
\end{center}

当输入错误时还可以按退格键删除一个字符。

\begin{center}
    \includegraphics[width=13cm]{image/testing/backspace}
    \fcaption{退格}
\end{center}

至此,我们实现的所有30条指令以及自己扩展的\texttt{ERET}指令均通过了测试。

\section{越过的千难万险}

\subsection{奇偶校验}

在单独调试串口的时候,助教早已告诉了我们,要开奇校验。由于我们一直没遇到问题,所以一直都没开奇校验。突然某一天,我们的串口突然不能正常工作了,连续两条串口发来的消息总会有一条出错。我们百思不得其解,我们仔细研究了其出错的情形,寻找出错数据的共同点,最后终于发现了原因。

\subsection{奇妙的A指令}

\subsection{无效的加速}



\renewcommand{\refname}{参考资料}
\begin{thebibliography}{9}
    \bibitem{} 刘卫东,李山山,宋佳兴. 计算机硬件系统实验教程. 清华大学出版社,2013.
    \bibitem{} \textit{Working with CORE Generator IP}.  \url{http://www.xilinx.com/support/documentation/sw_manuals/xilinx11/ise_c_using_coregen_ip.htm}.
\end{thebibliography}

\end{document}
% -----------------end------------------%
