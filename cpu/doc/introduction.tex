\section{概述}

我们实现了一个无延迟槽的带动态分支预测的流水线CPU。我们实现了一个通用性极佳(支持软硬件中断,支持绘图)的计算机。

我们最初的目的是尽可能不插,少插气泡,由于某些冲突必须暂停流水线,我们最终仅在3种情况插1个气泡,而绝大多数时候我们都无需插气泡。我们的CPU的主频最高可以达到21M(我们使用了ISE自带的模拟器件IP核DCM来分频)对于老师给的5个测例,我们花费的时间都非常少,因为我们几乎不用插气泡,所以花费的时间约等于指令数除以主频。

为了提高运行效率,我们增加了分支预测功能,branch指令和jump指令均在decode阶段进行,从而使得因为跳转引入的气泡尽可能的少。分支预测采取的是一个大小为3的查询表,每次使用pc进行查询,如果出现一次错误则更新,即记录上一次的结果。

除了CPU的核心以外,我们做了硬件中断,软件中断,像素映射的VGA接口的显示器(由于片内的RAM容量不大,不足以存下RGB,我们的显示是蓝白的)。

由于我们是像素映射,硬件的接口不仅单一而且不利于编程(非常繁琐,由于屏幕非常大,不足以用16位表示坐标,我们得传2次参数才能确定一个点),我们用软件实现了如下几个画图的接口:(详见第六节)

\begin{enumerate}
    \item 画一个点
    \item 画一条细线段
    \item 画一条粗线段
    \item 画一个等腰直角三角形
    \item 从数据RAM中读取一个形状(用于实现字符集,比如Unicode或者ASCII)
\end{enumerate}
